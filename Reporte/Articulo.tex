% This is samplepaper.tex, a sample chapter demonstrating the
% LLNCS macro package for Springer Computer Science proceedings;
% Version 2.20 of 2017/10/04
%
\documentclass{report}
%\documentclass[runningheads]{llncs}
%
\usepackage{graphicx}
\usepackage{algorithm}
\usepackage[spanish]{babel}
\usepackage[noend]{algpseudocode}
% Used for displaying a sample figure. If possible, figure files should
% be included in EPS format.
%
% If you use the hyperref package, please uncomment the following line
% to display URLs in blue roman font according to Springer's eBook style:
% \renewcommand\UrlFont{\color{blue}\rmfamily}

\begin{document}
%

\begin{titlepage}
\centering
{\bfseries\LARGE Instituto Nacional de Astrofísica Óptica y Electrónica \par}
\vspace{1cm}
{\scshape\Large Maestría en Ciencias Computacionales \par}
\vspace{3cm}
{\scshape\Huge Análisis Estadístico  \par}
\vspace{3cm}
{\itshape\Large Proyecto Estadística \par}
\vfill
{\Large Autor: \par}
{\Large Eliú Moreno Ramírez\par}
\vfill
{\Large Noviembre 2022 \par}
\end{titlepage}
           % typeset the header of the contribution
%

%
%
%
\section{Introducción}
El análisis de datos estadísticos es el proceso que nos permite interpretar los datos numéricos que disponemos, con el objetivo de tomar las decisiones más eficaces. De hecho,las empresas pueden tomar decisiones 5 veces más rápido que su competencia si las basan en el análisis de datos~\cite{ref_article0}.\\
Existen muchas herramientas para el análisis de datos estadísticos, Excel el cual es principal para oficinistas, sin embargo existen lenguajes de programación especializados para la estadística, aquellos usados por excelencia son \textit{R} el es un entorno de software libre (licencia GNU GLP) y lenguaje de programación interpretado, es decir, ejecuta las instrucciones directamente, sin una previa compilación del programa a instrucciones en lenguaje máquina~\cite{ref_article1}; y \textit{Python} la analítica de Python se refiere a aplicaciones de analítica avanzada que utilizan Python, un lenguaje de programación de código abierto. Python es uno de los lenguajes de codificación líderes en la actualidad para la analítica de datos con una amplia gama de casos de uso empresarial en diversas industrias~\cite{ref_article2}.


\begin{thebibliography}{8}
\bibitem{ref_article0} Piratoba Gil, R. P. Alarcón Guarín, Mg. R. (2011, octubre). Importancia de la estadística en una investigación cualitativa.
\bibitem{ref_article1}
Unir, V. (2020, 28 septiembre). Lenguaje R, ¿qué es y por qué es tan usado en Big Data? UNIR. https://www.unir.net/ingenieria/revista/lenguaje-r-big-data/
\bibitem{ref_article2}
¿Qué es la analítica Python? (s. f.). TIBCO Software. https://www.tibco.com/es/reference-center/what-are-python-analytics
\end{thebibliography}
\end{document}
